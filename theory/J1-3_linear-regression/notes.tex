\input ../../notes-header.tex

\begin{document}

\notetitle{02}{Linear Regression}

\textbf{Linear regression: the problem}
\begin{enumerate}
\item \fbox{Problem ($\times 6$)} We have a set of points $\{(x_i, y_i)\}$.
  Given a new $x$ value, we'd like to predict $\hat y$.
\item \textbf{Linear model:} We'll assume there exists a linear
  relationship $y=\theta_0 + \theta_1 x$ that offers a good
  approximation to the data.
\item In the real world, there's always noise
\item Sometimes other effects, too
\item Talk about meaning of slope
\item Dangers of extrapolation.  Example: global warming (a few data
  points in a few places at a few times)
\end{enumerate}

\fbox{Residuals ($\times 6$)}
\begin{enumerate}
\item \textit{résidu}
\item Goal: small residuals
\item Cost function: sum of squares of residuals
\item Residuals are what's left over after accounting for model fit.
\item A normal distribution of residuals is a good sign.  And
  conversely.
\item Not rules: rule of thumb.
\item Time series (\textit{une série temporelle}) often have important underlying structure.
  Correlation often doesn't model them well.
\end{enumerate}

\fbox{Outliers ($\times 8$)}
\begin{enumerate}
\item Points that fall farther from the regression line have more
  effect.  We call them \textit{high leverage} points.
\item If the effect is noticeable on the regression, we call it an
  \textit{influential point}.
\item If a point, omitted, would fall much further from the regression
  line, it is certainly influential.
\item If not enough data points, they might be all or mostly
  influential!
\item \fbox{Anscombe's quartet} --- summary statistics don't replace visualizing data
  % http://en.wikipedia.org/wiki/File:Anscombe\%27s_quartet_3.svg
  \begin{itemize}
  \item mean $x$ = 9
  \item variance = 11
  \item mean $y$ = 7.50
  \item sample variance $\in (4.122, 4.127)$
  \item $\corr(x,y) = 0.816$
  \item linear regression: $y=3 + x/2$
  \end{itemize}
\item Correlation does not imply \fbox{causation}---but it's a good hint
\end{enumerate}

\textbf{Linear regression}
\begin{enumerate}
\item Univariate --- 1 input, 1 continuous output
\item We think there's a linear model
\item Explanatory or predictor varaible
\item Response variable
\item $h_\theta(x) = \theta_0 + \theta_1 x$
\item \fbox{Cost function} : $J(\theta_0, \theta_1) = \frac{1}{2m} \sum_{i=1}^m (h_\theta(x_i) - y_i)^2$
\item Cost function = fonction objective?
\item $y$ vs $\hat{y}$
\item \fbox{Gradient descent ($\times 3$)} (algorithme du gradient)
\item Assignment is simultaneous
\item Outlier = \textit{donnée aberrante}
\end{enumerate}

\textbf{Linear algebra (review)}
\begin{enumerate}
\item Vector, \fbox{matrix}, transpose
\item addition, multiplication
\item vector space, basis vectors
\item linear transformation, $u=Av$, think about basis vectors
\item $A$, $A_{i,j}$
\end{enumerate}

\fbox{Notation used in machine learning}
\begin{enumerate}
\item $x_j^{(i)}$ --- value of feature $j$ in training sample $i$
\item $x^{(i)}$ --- training sample $i$
\item $m$ = number of training samples
\item $n = \mid x^{(i)} \mid$ = number of features
\item $x_0 = 1$ (often called bias)
\end{enumerate}

\textbf{Multiple regression}
\begin{enumerate}
\item Multiple explanatory variables, 1 continuous output
\item Fortunately, there are libraries to do this!
\end{enumerate}

\textbf{Other notes}
\begin{itemize}
\item Overfitting
\item Regularization (ridge regression, Tikhanov regularization): $-\lambda\sum \mbox{params}$
\item Polynomial regression
\item Gradient descent variants
  \begin{itemize}
  \item Batch gradient descent (all samples)
  \item Stochastic gradient descent (single sample each iteration) (faster for very large sets)
  \item Mini-batch gradient descent (several samples at each iteration)
  (sometimes smoother convergence than SGD, sometimes faster if software can parallelize)
  \item Coordinate gradient descent (one component each iteration)
  \item Note computational approximation if no derivative (and curse of dimensionality)
  \end{itemize}
\item When gradient descent doesn't work,
  \begin{itemize}
  \item plot the cost function over iterations
  \item if cost increasing or oscillating, reduce $\alpha$
  \item if leveled off, not much future gain
  \end{itemize}

\end{itemize}


\end{document}
