\input ../talk-header.tex
\title
{ML Week}
\subtitle{Natural Language Processing}

% If you wish to uncover everything in a step-wise fashion, uncomment
% the following command: 
%\beamerdefaultoverlayspecification{<+->}

\begin{document}

\begin{frame}
  \titlepage
\end{frame}

\begin{frame}
  \frametitle{Linear Programming}

    \begin{align*}
      \text{Maximize \ } & c^T x \\
      \text{subject to \ } & Ax \le b
    \end{align*}
\end{frame}

\begin{frame}
  \frametitle{Summarising Text}

  \only<1>{\begin{itemize}
    \item Abstractive \ \ \gray{(hard)}
    \item Extractive \ \ \gray{(select sentences)}
    \end{itemize}
  }
  \only<2>{
    Challenge problem (cf. greedy solutions):
    \begin{enumerate}
    \item[] The cat is in the kitchen.
    \item[] The cat drinks the milk.
    \item[] The cat drinks the milk in the kitchen.
    \end{enumerate}
  }

\end{frame}

\begin{frame}
  \frametitle{Summarising Text}

  \only<1>{
    \begin{itemize}
    \item Sentence selection
    \item Use n-grams
    \item Stemming
    \item Stop words
    \item Prune short sentences
    \end{itemize}
  }
  \only<2>{
    \blue{Outline:}
  \begin{itemize}
  \item ILP \ \ \gray{\it(optimisation linéaire en nombres entiers)}
  \item Maximum coverage model
  \end{itemize}
  }
  \only<3>{
    \blue{ILP in canonical form:}
    
    \begin{align*}
      \text{Maximize \ } & c^T x \\
      \text{subject to \ } & Ax \le b \\
      & x \ge 0 \\
      & x \in \mathbb{Z}^n
    \end{align*}
  }
  \only<4-6>{
    \blue{ILP in standard form:}
    
    \begin{align*}
      \text{Maximize \ } & c^T x \\
      \text{subject to \ } & Ax + s = b \\
      & s \ge 0 \\
      & x \in \mathbb{Z}^n
    \end{align*}
  }
  \only<5>{\blue{This is NP hard.}}
  \only<6>{\blue{Discussion: linear vs integer programming.}}
  \only<7>{
    Let
    \begin{align*}
      c_i \text{ : } & \text{ presence of concept } i \text{ in summary} \\
      w_i \text{ : } & \text{ weight associated with } c_i \\
      l_i \text{ : } & \text{ length of sentence } i \\
      s_j \text{ : } & \text{ presence of sentence } j \text{ in summary} \\
      L \text{ : } & \text{ summary length limit } \\
      {Occ}_{ij} \text{ : } & \text{ occurence of } c_i \text{ in } s_j \\
    \end{align*}
  }
  \only<8>{
    \blue{Summarisation}
    
    \begin{align*}
      \text{Maximize \ } & \sum_i w_i c_i \\
      \text{subject to \ } & \sum_j l_j s_j \le L \\
      & s_j {Occ}_{ij} \le c_i, \qquad \forall i, j \\
      & \sum_j s_j {Occ}_{ij} \ge c_i \qquad \forall i \\
      & c_j \in \{0,1\}, \qquad \forall i \\
      & s_j \in \{0,1\}, \qquad \forall j
    \end{align*}
  }
  \only<9>{
    Notes:
    \begin{itemize}
    \item Selecting a sentence selects all concepts it contains
    \item Selecting a concept requires it be in at least one sentence
    \item $ s_j {Occ}_{ij} \le c_i,\ \forall i, j \ \Rightarrow$
      no concept-less sentences
    \end{itemize}
  }
  
  \vfill
  \prevwork{Dan Gillick, Benoit Favre, \textit{A Scalable Global Model
      for Summarization}, 2009}
\end{frame}


\begin{frame}
  \frametitle{Sentiment Analysis}
  \only<1>{
    Many variations:
    \begin{itemize}
    \item Entire documents using computational linguistics
    \item Manually crafted lexicons
    \end{itemize}
  }
  \only<2>{
    Techniques
    \begin{itemize}
    \item Template instantiation\ \ \gray{(requires domain knowledge)}
    \item Passage extraction
    \end{itemize}
  }
\end{frame}

\begin{frame}
  \frametitle{Sentiment Analysis}
  \only<1-3>{
    \begin{itemize}
    \item     Extract ``opinion sentences'' based on the
      presence of a predetermined list of
      product features and adjectives.
      \only<2>{
        \begin{itemize}
        \item \blue{``The food is excellent.''}
        \item \blue{``The food is an excellent example of how not to cook.''}
        \end{itemize}
      }
    \item  Evaluate the sentences based on counts
      of positive vs negative polarity words (as
      determined by the Wordnet algorithm)
    \end{itemize}
    \only<3>{
      \blue{The good: fast, no training data, decent prediction.}

      \red{The bad: fails on multiple word sense, non-adjectives; sensitive to context.}
    }
    \vfill
    \prevwork{Hu and Lieu, Mining and Summarizing Customer Reviews, 2004}
  }
\end{frame}

\begin{frame}
  \frametitle{Sentiment Analysis}
  \only<1> {
    Words aren't enough.
    \begin{itemize}
    \item ``unpredictable plot'' vs ``unpredictable performance''
    \end{itemize} 
  }

  \prevwork{Turney, \textit{Thumbs Up or Thumbs Down? Semantic Orientation Applied to
      Unsupervised Classification of Reviews}, 2002}

\end{frame}


%%%%%%%%%%%%%%%%%%%%%%%%%%%%%%%%%%%%%%%%%%%%%%%%%%%%%%%%%%%%%%%%%%%%%%
%\talksection{Break}

\begin{frame}
  \frametitle{Questions?}
  \centerline{\large\url{purple.com/talk-feedback}}
\end{frame}

\end{document}
